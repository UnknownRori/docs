\documentclass[a4paper]{article}

\def\Author{UnknownRori}
\def\Title{Tugas Makalah Web 3.0}
\def\Lecture{Komunikasi Data}
\def\Campus{STMIK Amikom Surakarta}

\usepackage[utf8]{inputenc}
\usepackage[english]{babel}
\usepackage{graphicx}
\usepackage[dvipsnames]{xcolor}
\usepackage{datetime}
\usepackage{geometry}
\usepackage{fancyhdr}
\usepackage{lastpage}
\usepackage{hyphenat}
\usepackage[T1]{fontenc}
\usepackage{tgbonum}
\usepackage{lipsum}
\usepackage{hyperref}
\usepackage[document]{ragged2e}
\usepackage{array}

% Set the font family to Times New Roman
\fontfamily{ptm}

\geometry {
    a4paper,
    left=30mm,
    right=30mm,
}
\hypersetup {
      colorlinks=true,
      linkcolor=black,
      filecolor=black,
      urlcolor=black,
      pdftitle={\Title},
      pdfauthor={\Author},
      pdfsubject={Web 3.0},
      pdfproducer={Latex},
      pdfcreator={pdflatex},
}
\urlstyle{same}

\newdate{date}{15}{05}{2023}


\title{\Title}
\author{\Author}

\pagestyle{fancy}
\fancyfoot{}
\fancyfoot[RO]{\thepage/\pageref{LastPage}}
\fancyfoot[LF]{\textit{
\small
\Author
}}

\renewcommand{\footrulewidth}{0.4pt}

\pagestyle{fancy}

\begin{document}

\begin{center}
      \Huge
      \textbf{\Title}
      \linebreak
      \Large
      \textbf{\Lecture}
      \linebreak
      \linebreak
      \linebreak

      \includegraphics[width=10cm]{./Logo.png}
      \linebreak
      \linebreak
      \linebreak

      \Large
      Disusun Oleh :
      \linebreak
      Akbar Hendra Jaya   (2202030549)
      \linebreak
    %   Cantika Silvi A. S. (2202030550)
      \linebreak
      \linebreak
      \linebreak
      \linebreak

      \Huge
      \textbf{\Campus}
      \linebreak
      \normalsize
      \today
\end{center}

\pagebreak

% Make sure the size is normal again
\normalsize

\tableofcontents

\pagebreak

\section{Web 3.0}

\subsection{Mengapa dibuat Web 3.0}

\paragraph{
      \normalfont
      Konsep Web 3.0 muncul sebagai jawaban atas keterbatasan dan tantangan yang
      terkait dengan Web 2.0. Web 2.0, yang menjadi terkenal di awal tahun 2000-an,
      membawa kemajuan signifikan dalam konten buatan pengguna, media sosial, dan
      kolaborasi online. Namun, itu juga menyoroti kekurangan tertentu, termasuk
      masalah yang berkaitan dengan privasi data, sentralisasi, dan kontrol.
      \linebreak
      \linebreak
}

Perkembangan Web 3.0 dapat ditelusuri kembali ke beberapa faktor kunci dan
kemajuan teknologi :

\begin{enumerate}
      \item {
            Kekhawatiran atas kontrol terpusat
            }

      \item {
            Interoperabilitas dan solusi lintas rantai
            }

      \item {
            Penyimpanan dan komputasi terdesentralisasi
            }
\end{enumerate}

\subsection{Definisi Web 3.0}

\normalcolor

\paragraph{
      \normalfont
      Web 3.0 adalah generasi internet yang paling baru yang mentargetkan
      desentralisasi, keamanan, dan user-centric online experience dibanding
      dengan Web 2.0 yang sekarang ini. Web 2.0 yang merupakan kondisi internet
      saat ini, memiliki karakteristik platform dan jasa yang bergantung dengan
      sesuatu yang sentral seperti sosial media, \textit{Search Engines}, dan
      website \textit{E-Commerce}. Platform ini mengambil dan mengontrol data
      data pemakai dan pemakai biasanya memiliki kontrol yang terbatas terhadap
      data pribadinya.
}

\paragraph{
      \normalfont
      Di sisi lain Web 3.0, membayangkan ekosistem web yang lebih
      desentralisasi. Memakai teknologi seperti \textit{Blockchain}, jaringan
      desentralisasi (\textit{Decentralized Networks}) dan
      \textit{Cryptocurrencies} untuk menggeser dinamika kekuatan dan
      memungkinkan pengguna memiliki kontrol yang lebih besar dan kepemilikan
      data.
}

\paragraph{
      \normalfont
      Web 3.0 masih merupakan konsep yang berkembang dan realisasi penuh
      membutuhkan pengembangan teknologi, standar, dan adopsi yang meluas. Ini
      memiliki potensi untuk membentuk kembali bagaimana kita berinteraksi,
      bertransaksi dan membagi informasi di internet, memberdayakan individu dan
      membina lebih banyak lingkungan digital yang terbuka dan inklusif.
}


\subsection{Konsep dan Fitur Web 3.0}

Ada beberapa konsep dan fitur utama di Web 3.0, antara lain :

\begin{enumerate}
      \item{
                  \raggedleft\textbf{Desentralisasi (\textit{Decentralization})}
                  \linebreak
                  \justifying
                  Web 3.0 mentarget melakukan desentralisasi berbagai macam
                  aspek dalam internet, seperti mengurangi ketergantungan oleh
                  pihak sentral, seperti penyimpanan desentralisasi, komunikasi
                  dan komunikasi dan berbagai macam fungsionalitas lain.
            }

      \item {
            \raggedleft\textbf{Teknologi Blockchain (\textit{Blockchain Technology})}
            \linebreak
            \justifying
            \textit{Blockchain} merupakan pondasi untuk Web 3.0. Ini memungkinkan
            terciptanya internet yang aman, transparan, dan tahan rusak,
            memungkinkan kepercayaan dan verifikasi dalam berbagai interaksi
            online. Teknologi \textit{Blockchain} juga menfasilitasi aplikasi
            desentralisasi dan \textit{Smart Contracts}.
            }

      \item {
            \raggedleft\textbf{\textit{Smart Contracts}}
            \linebreak
            \justifying
            Web 3.0 memperkenalkan kontrak pintar, kontrak pinar ini dijalankan
            sendiri dengan aturan dan ketentuan yang telah ditentukan sebelumnya
            yang dikodekan didalamnya. \textit{Smart Contracts} mengotomatiskan
            pelaksanaan perjanjian dan transaksi, menghilangkan kebutuhan akan
            perantara dan meningkatkan efisiensi.
            }

      \item {
            \raggedleft\textbf{\textit{Cryptocurrencies} dan \textit{Tokenization}}
            \linebreak
            \justifying
            Web 3.0 Menggunakan mata uang kripto dan tokenisasi untuk
            mengaktifkan model ekonomi baru dan mendorong partisipasi pengguna.
            \textit{Cryptocurrencies} seperti \textit{Bitcoin} dan
            \textit{Ethereum} memainkan peran penting dalam menfasilitasi
            transaksi peer-to-peer dan mendukung aplikasi desentralisasi.
            }

      \item {
            \raggedleft\textbf{Meningkatkan Privasi dan Keamanan }
            \linebreak
            \justifying
            Web 3.0 menekankan privasi pengguna dan keamanan data. Melalui
            teknik kriptografi dan sistem terdesentralisasi, pengguna memiliki
            kendali lebih besar atas data pribadi mereka dan dapat memilih untuk
            membagikannya secara selektif atau anonim.
            }

      \item {
            \raggedleft\textbf{Interoperabilitas (\textit{Interoperability})}
            \linebreak
            \justifying
            Web 3.0 bertujuan untuk memungkinkan interoperabilitas yang mulus
            antara berbagai aplikasi, platform, dan \textit{Blockchain}. Hal ini
            memungkinkan pertukaran data, aset dan layanan yang efisiensi di
            berbagai jaringan, mendorong pengalaman web yang lebih terhubung dan
            kohesif.
            }
\end{enumerate}

\subsection{Manfaat dan Kekurangan Web 3.0}

Ada beberapa Manfaat dan Kekurangan Web 3.0

\subsubsection{Manfaat Web 3.0}

\begin{enumerate}
      \item {
            \raggedleft\textbf{Desentralisasi (\textit{Decentralization}) dan kontrol user}
            \linebreak
            \justifying
            Bertujuan untuk mengalihkan kekuasaan dari otoritas terpusat ke
            pengguna itu sendiri. Ini memungkinkan individu untuk memiliki
            kontrol lebih besar atas data, privasi, dan interaksi
            online mereka.
            }

      \item {
            \raggedleft\textbf{Meningkatkan privasi dan keamanan}
            \linebreak
            \justifying
            Web 3.0 menekankan langkah-langkah privasi dan keamanan yang lebih
            kuat, memanfaatkan teknik \textit{Cryptography} dan sistem
            terdesentralisasi. Ini dapat memberi pengguna kepercayaan diri yang
            lebih besar dalam aktivitas online mereka dan mengurangi risiko
            pelanggaran data dan akses tidak sah.
            }

      \item {
            \raggedleft\textbf{Kepercayaan dan Transparansi}
            \linebreak
            \justifying
            Penggunaan teknologi blockchain di Web 3.0 memungkinkan transaksi
            dan penyimpanan data yang transparan dan anti rusak. Ini menumbuhkan
            kepercayaan dan akuntabilitas, karena semua peserta dapat
            memverifikasi integritas informasi.
            }

      \item {
            \raggedleft\textbf{Model Ekonomi Baru}
            \linebreak
            \justifying
            Web 3.0 memperkenalkan mata uang kripto dan tokenisasi, memungkinkan
            model dan peluang ekonomi baru. Ini memungkinkan transaksi
            peer-to-peer langsung, memfasilitasi crowdfunding melalui penawaran
            koin awal (\textit{ICO}), dan memungkinkan aplikasi keuangan
            terdesentralisasi (\textit{DeFi}).
            }

      \item {
            \raggedleft\textbf{Interoperabilitas (\textit{Interoperability}) dan
                  \textit{Seamless Integration}}
            \linebreak
            \justifying
            Interoperabilitas dan integrasi tanpa batas: Web 3.0 bertujuan untuk menciptakan standar dan protokol yang memungkinkan interoperabilitas antara berbagai aplikasi dan jaringan. Hal ini memungkinkan integrasi tanpa batas
            dan pertukaran data, aset, dan layanan yang efisien di seluruh
            platform.
            }
\end{enumerate}

\subsubsection{Kekurangan Web 3.0}

\begin{enumerate}
      \item {
            \raggedleft\textbf{Tingginya kompleksitas}
            \linebreak
            \justifying
            Teknologi Web 3.0, seperti \textit{Blockchain} dan \textit{Smart
                  Contracts}, dapat menjadi rumit bagi pengguna rata-rata. Ada kurva
            pembelajaran yang terkait dengan pemahaman dan navigasi sistem dan
            alat baru ini.
            }

      \item {
            \raggedleft\textbf{Tantangan performa dan Skalabilitas}
            \linebreak
            \justifying
            Sistem berbasis \textit{Blockchain}, saat ini, menghadapi
            skalabilitas dan keterbatasan kinerja. Semakin banyak pengguna
            bergabung dengan jaringan dan volume transaksi meningkat, efisiensi
            jaringan \textit{Blockchain} dapat terpengaruh.
            }

      \item {
            \raggedleft\textbf{Regulasi dan Masalah Hukum}
            \linebreak
            \justifying
            Sifat terdesentralisasi dari Web 3.0 dapat menimbulkan tantangan
            dalam hal kepatuhan peraturan, kerangka hukum, dan tata kelola.
            Masalah yang terkait dengan yurisdiksi, verifikasi identitas, dan
            pencegahan penipuan dapat muncul.
            }

      \item {
            \raggedleft\textbf{Konsumsi Energi}
            \linebreak
            \justifying
            Beberapa jaringan \textit{Blockchain}, terutama yang menggunakan
            algoritma konsensus \textit{proof-of-work}, memerlukan daya
            komputasi dan konsumsi energi yang signifikan. Hal ini menimbulkan
            kekhawatiran tentang dampak lingkungan dari teknologi Web 3.0.
            }

      \item {
            \raggedleft\textbf{Kurangnya Standar}
            \linebreak
            \justifying
            Karena Web 3.0 masih berkembang, ada kekurangan standar dan praktik
            terbaik yang ditetapkan. Ini dapat menyebabkan fragmentasi, masalah
            interoperabilitas, dan tingkat adopsi yang lebih lambat.

            % Lack of established standards: As Web 3.0 is still evolving, there
            % is a lack of established standards and best practices. This can lead
            % to fragmentation, interoperability issues, and slower adoption
            % rates.
            }
\end{enumerate}

\pagebreak

\subsection{Perbedaan dengan Web 2.0}

\begin{center}
      \begin{tabular}{ || m{10em} | m{10em} | m{12em} || }
            \hline
            \multicolumn{1}{|| c | }{\textbf{Karakteristik}} & \multicolumn{1}{c}{\textbf{Web 2.0}}        & \multicolumn{1}{| c ||}{\textbf{Web 3.0}}                                                                            \\
            \hline\hline
            Kepusatan                                        & Centralisasi                                & Decentralisasi                                                                                                       \\
            \hline
            Privasi dan Kontrol Data                         & Pengguna memiliki kontrol kecil             & Pengguna memiliki kontrol                                                                                            \\
            \hline
            Kepercayaan                                      & Kepada pihak yang memberikan jasa           & Kepercayaan dengan pengguna lain dan transparan                                                                      \\
            \hline
            Model Ekonomi                                    & E-Commerce, Iklan, Data pengguna            & \textit{Crowdfunding}, \textit{peer-to-peer}, \textit{Decentralized Finance (DeFi) Applications}                     \\
            \hline
            Integrasi                                        & Terisolasi, dan kurangnya integrasi         & Mentargetkan menetapkan standar protokol memungkinkan efisiensi dalam pertukaran data                                \\
            \hline
            \textit{User Experience}                         & Fokus dalam \textit{User Generated Content} & Memperdayakan pengguna agar berpartisipasi, kepemilikan, dan kontrol yang lebih aktif atas pengalaman digital mereka \\
            \hline
            \textit{Status}                                  & Siap Pakai                                  & Masih dalam pembuatan                                                                                                \\
            \hline
      \end{tabular}
\end{center}
\subsection{Rangkuman}

Web 3.0 adalah generasi internet berikutnya yang bertujuan untuk menjadi lebih
terdesentralisasi, aman, dan berpusat pada pengguna dibandingkan dengan Web 2.0.
Ini memanfaatkan teknologi \textit{Blockchain}, \textit{Smart Contracts}, dan \textit{Cryptocurrency} untuk
mengalihkan kekuatan dari entitas terpusat ke pengguna. Web 3.0 menekankan
kontrol pengguna, privasi, dan kepemilikan data. Ini memperkenalkan model
ekonomi baru, memungkinkan integrasi dan interoperabilitas tanpa batas, dan
meningkatkan langkah-langkah keamanan. Web 3.0 adalah proses pengembangan
berkelanjutan dengan potensi untuk membentuk kembali internet dan menciptakan
ekosistem digital yang lebih terbuka dan digerakkan oleh pengguna.

\pagebreak

\section{Daftar Pustaka}

\begin{enumerate}
      \item \href{https://en.wikipedia.org/wiki/Web3}{\color{blue}https://en.wikipedia.org/wiki/Web3}
      \item \href{https://en.wikipedia.org/wiki/Web\_2}{\color{blue}https://en.wikipedia.org/wiki/Web\_2}
      \item \href{https://web3.foundation}{\color{blue}https://web3.foundation}
      \item \href{https://web3.foundation/about/}{\color{blue}https://web3.foundation/about/}
      \item ChatGPT
\end{enumerate}

\end{document}